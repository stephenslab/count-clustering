% adopt PLoS genetics environment settings
\documentclass[10pt,letterpaper]{article}
\usepackage[top=0.85in,left=2.75in,footskip=0.75in]{geometry}

% amsmath package, useful for mathematical formulas
\usepackage{amsmath}
%\usepackage{natbib}
% amssymb package, useful for mathematical symbols
\usepackage{amssymb}
\usepackage{booktabs}
\usepackage{xspace}
\usepackage{hyperref}
% graphicx package, useful for including eps and pdf graphics
% include graphics with the command \includegraphics
\usepackage{graphicx}


% cite package, to clean up citations in the main text. Do not remove.
\usepackage{cite}
\usepackage{caption}
\usepackage{subcaption}
\usepackage{rotating}

\usepackage{color} 

% Use doublespacing - comment out for single spacing
%\usepackage{setspace} 
%\doublespacing


% Text layout
\topmargin 0.0cm
\oddsidemargin 0.5cm
\evensidemargin 0.5cm
\textwidth 16cm 
\textheight 21cm

\setlength{\parskip}{1em}

% Bold the 'Figure #' in the caption and separate it with a period
% Captions will be left justified
\usepackage[labelfont=bf,labelsep=period,justification=raggedright]{caption}

% Use the PLoS provided bibtex style
\bibliographystyle{/Users/stephens/Dropbox/Documents/stylefiles/plos2009}

% Remove brackets from numbering in List of References
\makeatletter
\renewcommand{\@biblabel}[1]{\quad#1.}
\makeatother


% Leave date blank
\date{}

\pagestyle{myheadings}
%% ** EDIT HERE **
\usepackage{enumerate}
\usepackage{multirow} 
\usepackage{url}
\usepackage{xr} %for cross-referencing
%% ** EDIT HERE **
%% PLEASE INCLUDE ALL MACROS BELOW
\newtheorem{algorithm}{Algorithm}
\newtheorem{proposition}{Proposition}
\newtheorem{restateproposition}{Proposition}
\newtheorem{lemma}{Lemma}
\newtheorem{corollary}{Corollary}
\newtheorem{result}{Result}
\newtheorem{note}{Note}
\newtheorem{definition}{Definition}

\def\KL{\text{KL}}


% Text layout specific to Supplemental Materials
\topmargin 0.0cm
\oddsidemargin 0.5cm
\evensidemargin 0.5cm
\textwidth 16cm
\textheight 21cm

\setlength{\parskip}{1em}

\begin{document}

\paragraph*{S3 Fig.}
\label{figS3}
{\bf A comparison of ``accuracy" of hierarchical clustering vs. GoM on thinned GTEx data, with thinning parameters of $p_{thin}=0.01$ and $p_{thin}=0.001$.}  For each pair of tissue samples from the GTEx V6 data we assessed whether or not each clustering method (with $K=2$ clusters) separated the samples according to their tissue of origin, with successful separation indicated by a filled square. Thinning deteriorates accuracy compared with the unthinned data (Fig~2), but even then the model-based method remains more successful than the hierarchical clustering in separating the samples by tissue or origin.
 \begin{figure}[ht]
    \centering
     \begin{subfigure}[t]{0.5\textwidth}
        \centering
        \includegraphics[height=2.5in]{../../plots/hierarchical_separation_thinned_0_001.png}
        \caption{hierarchy thin 0.01}
    \end{subfigure}%
    ~
    \begin{subfigure}[t]{0.5\textwidth}
        \centering
        \includegraphics[height=2.5in]{../../plots/admixture_separation_thinned_0_01.png}
        \caption{GoM thin 0.01}
    \end{subfigure}\\

     \begin{subfigure}[t]{0.5\textwidth}
        \centering
        \includegraphics[height=2.5in]{../../plots/hierarchical_separation_thinned_0_001.png}
        \caption{hierarchy 0.001}
    \end{subfigure}%
    ~
    \begin{subfigure}[t]{0.5\textwidth}
        \centering
        \includegraphics[height=2.5in]{../../plots/admixture_separation_thinned_0_001.png}
        \caption{GoM thin 0.001}
    \end{subfigure}\\
\end{figure}

\end{document}
