\documentclass[10pt,letterpaper]{article}
\usepackage[top=0.85in,left=2.75in,footskip=0.75in]{geometry}

% amsmath package, useful for mathematical formulas
\usepackage{amsmath}
%\usepackage{natbib}
% amssymb package, useful for mathematical symbols
\usepackage{amssymb}
\usepackage{booktabs}
\usepackage{xspace}
\usepackage{hyperref}
% graphicx package, useful for including eps and pdf graphics
% include graphics with the command \includegraphics
\usepackage{graphicx}


% cite package, to clean up citations in the main text. Do not remove.
\usepackage{cite}
\usepackage{caption}
\usepackage{subcaption}
\usepackage{rotating}

\usepackage{color} 

% Use doublespacing - comment out for single spacing
%\usepackage{setspace} 
%\doublespacing


% Text layout
\topmargin 0.0cm
\oddsidemargin 0.5cm
\evensidemargin 0.5cm
\textwidth 16cm 
\textheight 21cm

\setlength{\parskip}{1em}

% Bold the 'Figure #' in the caption and separate it with a period
% Captions will be left justified
\usepackage[labelfont=bf,labelsep=period,justification=raggedright]{caption}

% Use the PLoS provided bibtex style
\bibliographystyle{/Users/stephens/Dropbox/Documents/stylefiles/plos2009}

% Remove brackets from numbering in List of References
\makeatletter
\renewcommand{\@biblabel}[1]{\quad#1.}
\makeatother


% Leave date blank
\date{}

\pagestyle{myheadings}
%% ** EDIT HERE **
\usepackage{enumerate}
\usepackage{multirow} 
\usepackage{url}
\usepackage{xr} %for cross-referencing
%% ** EDIT HERE **
%% PLEASE INCLUDE ALL MACROS BELOW
\newtheorem{algorithm}{Algorithm}
\newtheorem{proposition}{Proposition}
\newtheorem{restateproposition}{Proposition}
\newtheorem{lemma}{Lemma}
\newtheorem{corollary}{Corollary}
\newtheorem{result}{Result}
\newtheorem{note}{Note}
\newtheorem{definition}{Definition}

\def\KL{\text{KL}}


% Text layout specific to Supplemental Materials
\topmargin 0.0cm
\oddsidemargin 0.5cm
\evensidemargin 0.5cm
\textwidth 16cm
\textheight 21cm

\setlength{\parskip}{1em}


\begin{document}

\paragraph*{S2 Table.}
\label{supptab2}
{\bf Cluster Annotations of GTEx V6 Brain data with top driving gene summaries.}

\begin{table}[!hp]
\begin{adjustwidth}{-.5in}{0in}
\begin{tabular}{|p{0.7in}|p{0.7in}|p{1.4in}|p{3.6in}|}
\hline
Cluster & Top Driving \qquad Genes & Gene names & Gene Summary \\
\hline
 \multirow{3}{4em}{\small{1, Royal blue}}  &  \small{\textit{CLU}} &  \footnotesize{clusterin} & \scriptsize{protein encoded by this gene is a secreted chaperone that can under some stress conditions also be found in the cell cytosol, also involved in cell death, tumor progression, and neurodegenerative disorders.} \\
 					      & \small{\textit{OXT}} &  \footnotesize{oxytocin/neurophysin I prepropeptide} & \scriptsize{encodes a precursor protein that is processed to produce oxytocin and neurophysin I, involved in contraction of  smooth muscle during parturition and lactation, cognition, tolerance, adaptation and complex sexual and maternal behaviour.} \\
					      & \small{\textit{GLUL}} & \footnotesize{glutamate-ammonia ligase} & \scriptsize{catalyzes the synthesis of glutamine from glutamate and ammonia in an ATP-dependent reaction,  associated with congenital glutamine deficiency, and overexpression of this gene was observed in some primary liver cancer samples.} \\
\hline
 \multirow{3}{4em}{\small{2, Turquoise}}  & \small{\textit{ENC1}} & \footnotesize{ectodermal-neural cortex 1} & \scriptsize{plays a role in the oxidative stress response as a regulator of the transcription factor Nrf2, may play role in malignant transformation.} \\
 							&  \small{\textit{NCALD}} & \footnotesize{neurocalcin delta} & \scriptsize{ encodes a member of the neuronal calcium sensor (NCS), a regulator of G protein-coupled receptor signal transduction.}   \\
 					      & \small{\textit{YWHAH}} &  \footnotesize{tyrosine 3-monooxygenase/tryptophan 5-monooxygenase activation protein eta} & \scriptsize{mediate signal transduction by binding to phosphoserine-containing proteins, associated with early-onset schizophrenia and psychotic bipolar disorder.} \\
\hline
 \multirow{3}{4em}{\small{3, Lime green}} & \small{\textit{PKD1}} & \footnotesize{polycystin 1, transient receptor potential channel interacting} & \scriptsize{functions as a regulator of calcium permeable cation channels and intracellular calcium homoeostasis. It is also involved in cell-cell/matrix interactions and may modulate G-protein-coupled signal-transduction pathways.}\\
 					    & \small{\textit{CBLN3}} & \footnotesize{cerebellin 3 precursor} & \scriptsize{ contain a cerebellin motif and C-terminal C1q signature domain that mediates trimeric assembly of atypical collagen complexes} \\
					    &  \small{\textit{CHGB}} &  \footnotesize{chromogranin B} & \scriptsize{ encodes a tyrosine-sulfated secretory protein abundant in peptidergic endocrine cells and neurons. This protein may serve as a precursor for regulatory peptides.} \\
 \hline
  \multirow{3}{4em}{\small{4, Red}} & \small{\textit{PPP1R1B}} & \footnotesize{protein phosphatase 1 regulatory inhibitor sub-
unit 1B} & \scriptsize{encodes a bifunctional signal transduction molecule, may serve as a therapeutic target for neurologic and psychiatric disorders.}\\
 					    & \small{\textit{RGS14}} & \footnotesize{regulator of G-protein signaling 14} & \scriptsize{ attenuates the signaling activity of G-proteins, increases the rate of conversion of the GTP to GDP.} \\
					    &  \small{\textit{NCDN}} &  \footnotesize{neurochondrin} & \scriptsize{ encodes a leucine-rich cytoplasmic protein, essential for spatial learning processes.} \\
 \hline
 \multirow{3}{4em}{\small{5, Yellow orange}} & \small{\textit{MBP}} & \footnotesize{myelin basic protein} & \scriptsize{protein encoded is a major constituent of the myelin sheath of oligodendrocytes and Schwann cells in the nervous system.} \\
 					    & \small{\textit{GFAP}} & \footnotesize{glial fibrillary acidic protein} & \scriptsize{ encodes major intermediate filament proteins of mature astrocytes, a marker to distinguish astrocytes during development, mutations in this gene cause Alexander disease, a rare disorder of astrocytes in central nervous system.} \\
					    & \small{\textit{TF}}  & \footnotesize{transferrin}  & \scriptsize{transport iron from the intestine, reticuloendothelial system, and liver parenchymal cells to all proliferating cells in the body, involved in the removal of certain organic matter and allergens from serum.}\\
\hline
 \multirow{3}{4em}{\small{6, Yellow}} & \small{\textit{IQGAP1}} & \footnotesize{IQ motif containing GTPase activating protein 1} & \scriptsize{interacts with components of the cytoskeleton, with cell adhesion molecules, and with several signaling molecules to regulate cell morphology and motility.} \\
 					    & \small{\textit{A2M}} & \footnotesize{alpha-2-macroglobulin} & \scriptsize{ inhibits many proteases, including trypsin, thrombin and collagenase. A2M is implicated in Alzheimer disease (AD) due to its ability to mediate the clearance and degradation of A-beta, the major component of beta-amyloid deposits.} \\
					    & \small{\textit{C3}}  & \footnotesize{complement component 3}  & \scriptsize{plays a central role in the activation of complement system, associated with atypical hemolytic uremic syndrome and age-related macular degeneration in human patients.}\\
\hline
\end{tabular}
\end{adjustwidth}
\end{table}

\end{document}
