% adopt PLoS genetics environment settings
\documentclass[10pt,letterpaper]{article}
\usepackage[top=0.85in,left=2.75in,footskip=0.75in]{geometry}

% Template for PLoS
% Version 3.2 March 2016

% General commands for the entire paper
%
% Use Unicode characters when possible
\usepackage[utf8x]{inputenc}
% amsmath package, useful for mathematical formulas
\usepackage{amsmath}
%\usepackage{natbib}
% amssymb package, useful for mathematical symbols
\usepackage{amssymb}
\usepackage{booktabs}
\usepackage{xspace}
\usepackage{hyperref}
% graphicx package, useful for including eps and pdf graphics
% include graphics with the command \includegraphics
\usepackage{graphicx}

% Use adjustwidth environment to exceed column width (see example table in text)
\usepackage{changepage}

% textcomp package and marvosym package for additional characters
\usepackage{textcomp,marvosym}

% fixltx2e package for \textsubscript
\usepackage{fixltx2e}

% cite package, to clean up citations in the main text. Do not remove.
\usepackage{cite}
\usepackage{caption}
\usepackage{subcaption}
\usepackage{rotating}

\usepackage{color}

% Use doublespacing - comment out for single spacing
%\usepackage{setspace}
%\doublespacing

% Text layout
\topmargin 0.0cm
\oddsidemargin 0.5cm
\evensidemargin 0.5cm
\textwidth 16cm
\textheight 21cm

\setlength{\parskip}{1em}

% Bold the 'Figure #' in the caption and separate it with a period
% Captions will be left justified
\usepackage[labelfont=bf,labelsep=period,justification=raggedright]{caption}

% Use the PLoS provided bibtex style
\bibliographystyle{/Users/stephens/Dropbox/Documents/stylefiles/plos2009}

% Remove brackets from numbering in List of References
\makeatletter
\renewcommand{\@biblabel}[1]{\quad#1.}
\makeatother

% Use nameref to cite supporting information files (see Supporting Information section for more info)
\usepackage{nameref,hyperref}

% line numbers
\usepackage[right]{lineno}

% ligatures disabled
\usepackage{microtype}
\DisableLigatures[f]{encoding = *, family = * }

% Leave date blank
\date{}

\pagestyle{myheadings}
%% ** EDIT HERE **
\usepackage{enumerate}
\usepackage{multirow}
\usepackage{url}
\usepackage{xr} %for cross-referencing
%% ** EDIT HERE **
%% PLEASE INCLUDE ALL MACROS BELOW
\newtheorem{algorithm}{Algorithm}
\newtheorem{proposition}{Proposition}
\newtheorem{restateproposition}{Proposition}
\newtheorem{lemma}{Lemma}
\newtheorem{corollary}{Corollary}
\newtheorem{result}{Result}
\newtheorem{note}{Note}
\newtheorem{definition}{Definition}

\def\KL{\text{KL}}


% Text layout
\raggedright
\setlength{\parindent}{0.5cm}
\textwidth 5.25in
\textheight 8.75in

% Bold the 'Figure #' in the caption and separate it from the title/caption with a period
% Captions will be left justified
\usepackage[aboveskip=1pt,labelfont=bf,labelsep=period,justification=raggedright,singlelinecheck=off]{caption}
\renewcommand{\figurename}{Fig}

%------ bibliography
% Use the PLoS provided BiBTeX style
\bibliographystyle{plos2015}
% Remove brackets from numbering in List of References
\makeatletter
\renewcommand{\@biblabel}[1]{\quad#1.}
\makeatother


% Header and Footer with logo
\usepackage{lastpage,fancyhdr,graphicx}
\usepackage{epstopdf}
\pagestyle{myheadings}
\pagestyle{fancy}
\fancyhf{}
\setlength{\headheight}{27.023pt}
\lhead{\includegraphics[width=2.0in]{PLOS-submission.eps}}
\rfoot{\thepage/\pageref{LastPage}}
\renewcommand{\footrule}{\hrule height 2pt \vspace{2mm}}
\fancyheadoffset[L]{2.25in}
\fancyfootoffset[L]{2.25in}
\lfoot{\sf PLOS}

%% Include all macros below

\newcommand{\lorem}{{\bf LOREM}}
\newcommand{\ipsum}{{\bf IPSUM}}

%% END MACROS SECTION

%% Author's settings
\def\KL{\text{KL}}


<<<<<<< HEAD
=======
% Text layout specific to Supplemental Materials
\topmargin 0.0cm
\oddsidemargin 0.5cm
\evensidemargin 0.5cm
\textwidth 16cm
\textheight 21cm

\setlength{\parskip}{1em}

>>>>>>> 12c8c624449cc2a69a567ae31d354e2591170106
\begin{document}

\paragraph*{S9 Fig.}

\label{figS9}
<<<<<<< HEAD
{\bf Comparison between GoM model and hierarchical in terms of power to separate samples from pairs of tissues.
\begin{figure}[ht]
\centering
\includegraphics[height=6.3in, width=7in]{../../plots/gtex_hierarchical.jpeg}
 \caption{A comparison of accuracy of GoM model vs hierarchical clustering.  Image plots to compare the GoM model with 4 different hierarchical clustering models on various transformations of the data. For each pair of tissues from the GTEx data we assessed whether or not each method (with $K=2$ clusters) separated the samples precisely according to their actual tissue of origin, with successful separation indicated by a filled square. Very clearly, the GoM model seems to be more successful in separating pairs of tissues compared to any of the hierarchical clustering approaches. In SubFig (a), hierarchical clustering was performed on log counts per million (cpm) data using Euclidean distance. In SubFig(b), the log cpm data data was mean and scale transformed for each gene and then the hierarchical clustering was performed on the transformed data using the Euclidean distance. In SubFig (d), the hierarchical clustering was  performed on counts data with the assumption the counts $c_{ng}$ for each gene have a variance  $\bar{c}_{g} + 1$, which we used to scale while computing distance matrix. In SubFig (e), we took the same scaled data as in SubFig(c), but we additionally performed mean and scale adjustments further so that all genes have expression of mean 0 and variance 1. In SubFig(c), GoM model is used to separate the tissues. Very clearly, GoM model seems to be performing better than any of the hierarchical methods. 
 }
\end{figure}

\end{document}
=======
{\bf Mouse embryo single cell sample visualization of GoM results with K=6 including
(a) genes that are located on the odd-numbered chromosomes,
and (b) genes that are located on the even-numbered chromosomes.}
Single cell samples collected at the same developmental stage are represented by points of matching color.
\begin{figure*}[ht]
\centering
\includegraphics[height=6in, width=5in]{../../src/figure/deng-chromosome.Rmd/deng-chromosome-plot-1.png}
\end{figure*}

\end{document}
>>>>>>> 12c8c624449cc2a69a567ae31d354e2591170106
