% adopt PLoS genetics environment settings
\documentclass[10pt,letterpaper]{article}
\usepackage[top=0.85in,left=2.75in,footskip=0.75in]{geometry}

% Template for PLoS
% Version 3.2 March 2016

% General commands for the entire paper
%
% Use Unicode characters when possible
\usepackage[utf8x]{inputenc}
% amsmath package, useful for mathematical formulas
\usepackage{amsmath}
%\usepackage{natbib}
% amssymb package, useful for mathematical symbols
\usepackage{amssymb}
\usepackage{booktabs}
\usepackage{xspace}
\usepackage{hyperref}
% graphicx package, useful for including eps and pdf graphics
% include graphics with the command \includegraphics
\usepackage{graphicx}

% Use adjustwidth environment to exceed column width (see example table in text)
\usepackage{changepage}

% textcomp package and marvosym package for additional characters
\usepackage{textcomp,marvosym}

% fixltx2e package for \textsubscript
\usepackage{fixltx2e}

% cite package, to clean up citations in the main text. Do not remove.
\usepackage{cite}
\usepackage{caption}
\usepackage{subcaption}
\usepackage{rotating}

\usepackage{color}

% Use doublespacing - comment out for single spacing
%\usepackage{setspace}
%\doublespacing

% Text layout
\topmargin 0.0cm
\oddsidemargin 0.5cm
\evensidemargin 0.5cm
\textwidth 16cm
\textheight 21cm

\setlength{\parskip}{1em}

% Bold the 'Figure #' in the caption and separate it with a period
% Captions will be left justified
\usepackage[labelfont=bf,labelsep=period,justification=raggedright]{caption}

% Use the PLoS provided bibtex style
\bibliographystyle{/Users/stephens/Dropbox/Documents/stylefiles/plos2009}

% Remove brackets from numbering in List of References
\makeatletter
\renewcommand{\@biblabel}[1]{\quad#1.}
\makeatother

% Use nameref to cite supporting information files (see Supporting Information section for more info)
\usepackage{nameref,hyperref}

% line numbers
\usepackage[right]{lineno}

% ligatures disabled
\usepackage{microtype}
\DisableLigatures[f]{encoding = *, family = * }

% Leave date blank
\date{}

\pagestyle{myheadings}
%% ** EDIT HERE **
\usepackage{enumerate}
\usepackage{multirow}
\usepackage{url}
\usepackage{xr} %for cross-referencing
%% ** EDIT HERE **
%% PLEASE INCLUDE ALL MACROS BELOW
\newtheorem{algorithm}{Algorithm}
\newtheorem{proposition}{Proposition}
\newtheorem{restateproposition}{Proposition}
\newtheorem{lemma}{Lemma}
\newtheorem{corollary}{Corollary}
\newtheorem{result}{Result}
\newtheorem{note}{Note}
\newtheorem{definition}{Definition}

\def\KL{\text{KL}}


% Text layout
\raggedright
\setlength{\parindent}{0.5cm}
\textwidth 5.25in
\textheight 8.75in

% Bold the 'Figure #' in the caption and separate it from the title/caption with a period
% Captions will be left justified
\usepackage[aboveskip=1pt,labelfont=bf,labelsep=period,justification=raggedright,singlelinecheck=off]{caption}
\renewcommand{\figurename}{Fig}

%------ bibliography
% Use the PLoS provided BiBTeX style
\bibliographystyle{plos2015}
% Remove brackets from numbering in List of References
\makeatletter
\renewcommand{\@biblabel}[1]{\quad#1.}
\makeatother


% Header and Footer with logo
\usepackage{lastpage,fancyhdr,graphicx}
\usepackage{epstopdf}
\pagestyle{myheadings}
\pagestyle{fancy}
\fancyhf{}
\setlength{\headheight}{27.023pt}
\lhead{\includegraphics[width=2.0in]{PLOS-submission.eps}}
\rfoot{\thepage/\pageref{LastPage}}
\renewcommand{\footrule}{\hrule height 2pt \vspace{2mm}}
\fancyheadoffset[L]{2.25in}
\fancyfootoffset[L]{2.25in}
\lfoot{\sf PLOS}

%% Include all macros below

\newcommand{\lorem}{{\bf LOREM}}
\newcommand{\ipsum}{{\bf IPSUM}}

%% END MACROS SECTION

%% Author's settings
\def\KL{\text{KL}}


% Text layout specific to Supplemental Materials
\topmargin 0.0cm
\oddsidemargin 0.5cm
\evensidemargin 0.5cm
\textwidth 16cm
\textheight 21cm

\setlength{\parskip}{1em}

\begin{document}

\paragraph*{S19 Fig.}

\label{figS19}
{\bf Comparison between GoM model and hierarchical clustering under different scenarios of data transformation.} We used GTEx V6 data for model performance comparisons. Specifically, for every pair of the 53 tissues, we assessed the ability of the methods to separate samples according to their tissue of origin. The subplots of heatmaps show the results of evaluation under different scenarios. Filled squares in the heatmap indicate successful separation of the samples in corresponding tissue pair comparison. (A) Hierarchical clustering on log2 counts per million (CPM) transformed data using Euclidean distance. (B) Hierarchical clustering on the standardized log2-CPM transformed data (transformed values for each gene was mean and scale transformed) using the Euclidean distance. (D) Hierarchical clustering on counts data with the assumption that, for each gene the sample read count $c_{ng}$ has a variance $\bar{c}_{g} + 1$ that is constant across samples. And, the the gene-specific variance $\bar{c}_{g} + 1$ was used to scale the distance matrix for clustering. (C) GoM model of $K = 2$ applied to counts. (E) Hierarchical clustering applied to adjusted count data. Each gene has a mean expression value of 0 and variance of 1. Taken together, these results suggest that regardless of the different data transformation scenarios, the GoM model with $K = 2$ is able to separate samples of different tissue of origin, better than hierarchical cluster methods.
\begin{figure}[ht]
\centering
\includegraphics[height=6in, width=7in]{../../plots/gtex_hierarchical.jpeg}
\end{figure}

\end{document}
